\documentclass[ngerman,12pt,a4paper
,pdftex]{article}

\usepackage{amsmath}
\usepackage[utf8]{inputenc}
\usepackage{babel}
\usepackage[OT1]{fontenc}
\usepackage{graphicx}
\usepackage{framed}
\usepackage{blindtext}
\usepackage{setspace}
\usepackage{listings}
\usepackage{color}
\usepackage{hyperref}
\usepackage{lipsum}
\usepackage{cleveref}
\usepackage[printonlyused]{acronym}
\usepackage[center]{caption}
\usepackage{tocbibind}
\usepackage{titlesec}
\usepackage{fancyhdr}
\usepackage{tocloft}
\usepackage{latexsym}              
\usepackage[mathletters]{ucs}   
\usepackage[a4paper,hmargin=2.5cm,bottom=3cm]{geometry}
\usepackage{amssymb}
\usepackage{amsthm}
\usepackage{multicol}
\renewcommand\cftbeforetoctitleskip{-2cm}        % -> To fit TOC on one page

\title{Zusammenfassung Analysis 1 \\[1ex] \large Karlsruher Institut für Technologie \\[1ex] Wintersemester 2023/2024}
\author{Niklas Rodenbüsch}
\date{}

\newcommand{\cf}[1]{\[#1\]}
\newcommand{\f}[1]{$#1$}

\newtheoremstyle{nodot}
  {\topsep}   % Space above
  {\topsep}   % Space below
  {}          % Body font
  {}          % Indent amount
  {\bfseries} % Theorem head font
  {}          % Punctuation after theorem head
  { }         % Space after theorem head
  {}          % Theorem head spec (can be left empty, meaning `normal`)

% Define theorem envirnment
\theoremstyle{nodot}
\newtheorem*{definition}{Definition:}

\theoremstyle{nodot}
\newtheorem*{trick}{Trick:}

\begin{document}


    \pagestyle{fancy}
    \fancyhf{}
    \fancyhead[L]{\nouppercase{\leftmark}}
    \fancyfoot[C]{\thepage}
    \renewcommand{\headrulewidth}{0.2pt}

    \maketitle
    \newpage
    \pagenumbering{Roman}
    

    \begin{onehalfspace}
        \tableofcontents
        \newpage
        \newcounter{savepage}
        \setcounter{savepage}{\value{page}}
        \pagenumbering{arabic}
        \newpage

        \section{Grundlagen}
        % \subsection*{Allgemeines}
% \begin{itemize}
%     \item \f{sin(x\pm y) = sin(x)\cdot cos(y) \pm cos(x)\cdot sin(y)}\\
%     \f{cos(c\pm y) = cos(x)\cdot cos(y) \mp sin(x)\cdot sin(y)}
% \end{itemize}

\subsection*{Abbildungen}
Sei \f{f:A \rightarrow B}. Es gilt:
\begin{itemize}
    \item f ist \textbf{injektiv}, wenn: \f{\forall a_1, a_2 \in A : f(a_1) = f(a_2) \Rightarrow a_1 = a_2}
    \item f ist \textbf{surjektiv}, wenn: \f{\forall b \in B\quad  \exists a \in A : f(a) = b}
    \item f ist \textbf{bijektiv}, wenn f injektiv und surjektiv ist
\end{itemize}

\subsection*{Relationen}
Sei \f{x,y,z \in M} und \f{R} eine Relation:
\begin{itemize}
    \item \textbf{Reflexiv}, wenn \f{xRx}
    \item \textbf{Symmetrisch}, wenn \f{xRy \Leftrightarrow yRx}
    \item \textbf{Antisymmetrisch}, wenn \f{xRy \wedge yRz \Rightarrow x=y}
    \item \textbf{Transitiv}, wenn \f{xRy \wedge yRz \Rightarrow xRz}
    \item \textbf{Äquivalent}, wenn \f{R} reflexiv, symmetrisch und transitiv ist.\\
\end{itemize}
Sei \f{\sim } eine Äquivalenzrelation auf \f{M}.
\begin{itemize}
    \item \f{\left[x\right] _{\sim} = \left\{y\in M | x \sim y\right\}} ist \textbf{Äquivalenzklasse}
    \item \f{M/ \sim = \left\{\left[x\right] _{\sim} | x \in M\right\} }
    \item \f{\mathbb{Z} /n\mathbb{Z} = \left\{\left[0\right],\left[1\right], ...,\left[n-1\right] \right\}}
\end{itemize}
        % \subsection{Aussagenlogik}
        % \begin{definition}
        %     Eine \textit{Aussage} ist eine wohlgeformte mathematische Behauptung, die entweder \textit{wahr} oder \textit{falsch} ist.
        % \end{definition}
        \section{Komplexe Zahlen \f{\mathbb{C}}}
        % \begin{itemize}
%     \item[$\diamond $] \bfseries{: 9.1a-b, 9.4, 12.1}
% \end{itemize}
\begin{multicols}{2}
    \begin{itemize}
        \item \f{\mathbb{C} := { \left\{x+yi : x,y\in  \mathbb{R}\right\}}}
        \item \f{|z|=|x+yi|=\sqrt{x^2+y^2}}
        \item \f{i^0=1}, \f{i^1=i}, \f{i^2=-1}, \f{i^3=-i}
        \item \f{|w\cdot z| = |w|\cdot |z|}
        \item \f{|w + z| \leqslant |w| + |z|}
        \item \f{w = x+yi \Leftrightarrow \overline{w}=x-yi }
    \end{itemize}
\end{multicols}
\begin{itemize}
    \item \f{(x_1+y_1i)+(x_2+y_2i)=(x_1+x_2)+(y_1+y_2)i}
    \item \f{(x_1+y_1i)\cdot(x_2+y_2i)=(x_1x_2-y_1y_2)+(x_1y_2+x_2y_1)i}
\end{itemize}
        \section{Vollständige Induktion}
        \begin{enumerate}
    \item Zu beweisende Aussage (1) aufschreiben\\
    \f{\rightarrow} "`Sei \f{A(n)} die Aussage (1)"'
    \item \textbf{IA:} "`Es gilt \f{A(1)}, da ..."'
    \item \textbf{IV:} "`Sei \f{n\in \mathbb{N}}, es gelte \f{A(n)}"'
    \item \textbf{IS:} "`Für \f{A(n+1)} gilt ... \f{\overset{IV}{=}} ..."'
\end{enumerate}

        \section{Folgen \& Grenzwerte}
        \subsection{Allgemein}
\begin{itemize}
    \item Schreibweise: \f{(a_n)_{n \in \mathbb{N}}}
    \item Teilfolge von \f{(a_n)_n: (a_{n_k})_k}
    \begin{itemize}
        \item Jede Folge reeller Zahlen besitzt eine monotone Teilfolge.
        \item Jede beschränkte Folge reeller Zahlen besitzt eine konvergente Teilfolge.
    \end{itemize}    
    \item Beschränktheit: \f{(x_n)} ist beschränkt falls \f{\exists c>0 \textrm{ so dass } |x_n|\leq c\quad \forall n\in\mathbb{N}}
    \item Monotonie: \f{(x_n)} ist:
    \begin{itemize}
        \item monoton fallend: \f{\forall n : \frac{x_{n+1}}{x_n} \leq 1}
        \item monoton wachsend: \f{\forall n : \frac{x_{n+1}}{x_n} \geq 1}
    \end{itemize}
    \item \f{a} ist Häufungspunkt von \f{(a_n)_n}. \f{\Leftrightarrow \exists (a_{n_k})_k: a_{n_k} \overset{k \to \infty}{\longrightarrow} a}
    \item \f{(a_n)_n} monoton und beschränkt \f{\Rightarrow (a_n)_n} konvergent.
    \item \f{(a_n)_n, (b_n)_n} konvergent:
    \begin{itemize}
        \item \f{(a_n\dotplus b_n)_n} konvergent, \f{\lim{(a_n\dotplus b_n)}=\lim{(a_n)}\dotplus \lim{(b_n)}}
        \item \f{\lim(b_n)\neq 0 \Rightarrow \exists N \in \mathbb{N}: b_n \neq 0 \quad \forall n \geq N}, \f{(\frac{a_n}{b_n})_{n\geq N}} konvergent, \\
        \f{\lim{(\frac{a_n}{b_n})} = \frac{\lim(a_n)}{\lim(b_n)}}
        \item \f{a_n \leq b_n \quad \forall n \in \mathbb{N} \Rightarrow \lim(a_n) \leq \lim(b_n)}
        \item \f{\lim_{n\to \infty}|x_n|=|x|}
        \item Sandwich-Kriterium:\\
        \f{\lim_{n\to\infty}(a_n)=\lim_{n\to\infty}(b_n)=x} und \f{a_n\leq c_n\leq b_n \Rightarrow \lim_{x\to\infty}(c_n)=x}
    \end{itemize}
\end{itemize}

\subsection{Bekannte Folgen}
\begin{itemize}
    \item \f{\lim_{n \to \infty}(1\pm \frac{1}{n})^n = e^{\pm 1}}
    \item \f{\lim_{n \to \infty}\sqrt[n]{a}=1} für \f{a \in \mathbb{R}^+\cup \left\{\infty\right\}}
    \item \f{(a_n)_{n\in \mathbb{N}}} ist \textbf{Cauchy-Folge}, wenn \f{\forall \varepsilon > 0.} \f{\exists N \in \mathbb{N}.} \f{\forall m,n \geq N: |a_m-a_n|<\varepsilon}\\
    Reelle Cauchy-Folgen konvergieren immer.
\end{itemize}
        \section{Reihen \& Konvergenz}
        \subsection{Allgemein}
\begin{itemize}
    \item Schreibweise: Reihe \f{A_n=\sum_{k=n_0}^{^n}a_k,\quad (a_k)_k} ist "`zugehörige"' Folge
    \item \f{A_n} konvergent: \f{\sum_{k=n_0}^{n}a_k=\lim(A_n)}
    \item \f{\sum_{k=n_0}^{n}a_k} konvergent \f{\Rightarrow a_n \overset{n \to \infty}{\longrightarrow} 0}
    \item Die Summe zweier konvergenter Reihen ist konvergent. Eine konvergente Reihe bleibt bei Multiplikation mit einem reellen Skalar konvergent.
    \item \f{\sum a_n} absolut konvergent \f{:\Leftrightarrow\sum |a_n|} konvergent
    \item \f{\sum a_n} absolut konvergent \f{\Rightarrow \sum a_n} konvergent, \f{\quad|\sum a_n|\leq\sum|a_n|}
\end{itemize}

\subsection{Kriterien}
Sei die Reihe \f{A_n := \sum_{n=1}^{\infty}a_n}.
\begin{enumerate}
    \item \textbf{Nullfolge:}\\ 
    Für die Konvergenz muss \f{\lim_{n \to \infty}a_n = 0} gelten. Ist \f{a_n} keine Nullfolge, folgt daraus direkt Divergenz.
    \item \textbf{Cauchy's Konvergenzkriterium:}\\
    \f{A_n} konvergent \f{\Leftrightarrow \forall\varepsilon>0 \quad \exists N\in\mathbb{N}: |\sum_{k=m}^{n}a_k|<\varepsilon\quad\forall m,n \geq N}\\
    Gilt nur für \f{\mathbb{R}, \mathbb{C}}, nicht aber für \f{\mathbb{Q}}
    \item \textbf{Quotientenkriterium:}\\
    Sei \f{\lim_{n \to \infty}|\frac{a_{n+1}}{a_n}| =L}.\\
    Die Reihe konvergiert absolut, wenn \f{L < 1} und divergiert wenn \f{L > 1}.
    \item \textbf{Wurzelkriterium:}\\
    Sei \f{\lim_{n \to \infty}\sqrt[n]{|a_n|}=L}. \\
    Die Reihe konvergiert absolut, wenn \f{L < 1} und divergiert wenn \f{L > 1}.
    \item \textbf{Majorantenkriterium:}\\
    Sei die Reihe \f{\sum_{n=1}^{\infty}b_n} konvergent. Wenn \f{|b_n| \geq |a_n| \geq 0} für alle \f{n \geq n_0}, dann konvergiert auch die Reihe \f{\sum_{n=1}^{\infty}a_n} (absolut).
    \item \textbf{Minorantenkriterium:}\\
    Sei die Reihe \f{\sum_{n=1}^{\infty}b_n} divergent. Wenn \f{0 \leq b_n \leq a_n} für alle \f{n \geq n_0}, dann divergiert auch die Reihe \f{\sum_{n=1}^{\infty}a_n}.
    \item \textbf{Leibniz-Kriterium (alternierende Reihen):}\\
    Eine alternierende Reihe \f{\sum_{n=1}^{\infty}(-1)^na_n} konvergiert, wenn die folgenden Bedingungen erfüllt sind:
    \begin{enumerate}
        \item \f{0\leq a_{n+1}\leq a_n} für alle \f{n} (monoton fallend),
        \item \f{\lim_{n \to \infty}a_n=0} (Nullfolge)
    \end{enumerate}
    \item \textbf{Umordnungsgesetz:}\\
    \f{A_n} absolut konvergent \f{\Rightarrow} jede Umordnung der Reihe konvergiert gegen \f{A}
\end{enumerate}

\subsection{Bekannte Reihen}
\begin{itemize}
    \item \textbf{(Allgemeine) Harmonische Reihe:}
    \begin{itemize}
        \item \f{\sum_{k=1}^{n}\frac{1}{k} \to} divergent
        \item \f{\sum_{k=1}^{n}\frac{1}{k^\alpha} \to} konvergent für \f{\alpha > 1}
    \end{itemize}
    \item \textbf{Geometrische Reihe:}
    \begin{itemize}
        \item \f{|q| < 1 \Rightarrow\sum_{k=1}^{n}q^k=\frac{1}{1-q} \to} konvergent
        \item \f{|q| \geq 1 \Rightarrow\sum_{k=1}^{n}q^k \to} divergent
    \end{itemize}
    \item \f{\sum_{k=0}^{\infty}\frac{1}{k!} \to} konvergent \f{(=e)}
\end{itemize}

\newpage
\subsection{Potenzreihen}
\begin{itemize}
    \item Sei \f{(a_n)} ein Folge in \f{\mathbb{C}} und \f{z \in \mathbb{C}}, dann ist \f{P(z) := \sum_{n=1}^{\infty}a_nz^n} eine Potenzreihe.
    \item Konvergenzradius \f{R := \limsup_{n\to\infty}\frac{1}{\sqrt[n]{|a_n|}} }\\
    Falls \f{\lim_{n\to\infty}|\frac{a_n}{a_{n+1}}|} existiert, dann gilt: \f{R = \lim_{n\to\infty}|\frac{a_n}{a_{n+1}}|}
    \begin{itemize}
        \item \f{|z|<R \Rightarrow P(z)} konvergiert
        \item \f{|z|>R \Rightarrow P(z)} divergiert
        \item \f{|z|=R \Rightarrow} beides kann passieren
    \end{itemize}
    \item Verhalten am Rand des Konvergenzradius:\\
    Sei der Konvergenzradius \f{R > 0} der Potenzreihe \f{\sum_{n=1}^{\infty}a_nx^n} gegeben. Falls gefragt wird, wie sich die Reihe am Rand des Konvergenzkreises verhält, so sollte man \f{x = -R} und \f{x = R} auf Konvergenz überprüfen. Da \f{R = |\varrho |}:\\
    \f{\sum_{n=1}^{\infty}a_n(-R)^n = \sum_{n=1}^{\infty}a_n(-1)^nR^n\quad \Rightarrow } Leibnizkriterium verwenden.\\
    \textbf{Tipp:} Überprüfen, ob \f{a_n} eine Nullfolge ist.
\end{itemize}
        % \section{Potenzreihen}
        % \lipsum[1]
        %\section{Stetigkeit}
        \section{Stetige Funktionen}
        \begin{itemize}
    \item Funktion \f{f} definiert als: \f{\quad f: \underbrace{X}_\textrm{Definitionsbereich} \to \underbrace{Y}_\textrm{Bildbereich}}
    \item Eigenschaften von Funktionen:
    \begin{itemize}
        \item Injektivität: \f{f(x_1)=f(x_2)\Rightarrow x_1=x_2}
        \item Surjektivität: \f{\forall y\in Y: \exists x\in X: f(x) = y}
        \item Bijektivität: \f{f \textrm{ ist bijektiv} \Leftrightarrow f \textrm{ ist surjektiv und injektiv}}
        \item Umkehrfunktion: \f{f \textrm{ ist bijektiv} \Rightarrow \exists f^{-1}:f^{-1}(y)=x\quad \forall x\in X, \forall y \in Y}
        \item Beschränktheit: \f{f} heißt beschränkt, falls \f{\exists c>0} so dass \f{\forall x \in X: |f(x)|<c}
    \end{itemize}
    \item \textbf{Zwischenwertsatz:} Sei \f{f:\left[a,b\right]\to\mathbb{R}} eine stetige und reelle Funktion. Es gilt:\\
    \f{f(a)\leq\gamma \leq f(b) \Rightarrow \exists c \in \left[a,b\right]:f(c)=\gamma }
    % \item Die Exponentialfunktion (\f{e^x}) wächst schneller als Polynome \f{(z^n, \textrm{ mit } n\neq z)}. Es gilt: 
\end{itemize}


\subsection{Stetigkeit}
\begin{itemize}
    \item Definition durch \f{\varepsilon\delta -}Kriterium:\\
    \f{f} heißt stetig in \f{x_0} falls \f{\forall\varepsilon>0.\quad\exists\delta>0, \textrm{ sodass } \forall x\in X,} gilt:\\ \f{|x-x_0|<\delta \Rightarrow |f(x)-f(x_0)|<\varepsilon}.\\
    Weiter heißt \f{f} stetig, wenn \f{f} stetig in \f{x_0} \f{\forall x_0\in X} ist.\\
    \f{\delta} ist in Abhängigkeit von \f{\varepsilon} und \f{x_0} zu wählen.
    \item Definition durch Grenzwert:\\
    \f{f} heißt stetig in \f{x_0}, wenn \f{\lim_{x \to x_0}f(x) \textrm{ existiert und } \lim_{x \to x_0}f(x)= f(x_0)}.
    \item Seien \f{f,g} stetige Funktionen:
    \begin{multicols}{2}
        \begin{itemize}
            \item \f{f \circ g} ist stetig.
            \item \f{f\dotplus g} ist stetig.
            \item \f{|f|, \overline{f}, \mathfrak{R}(f), \mathfrak{J}(f)} sind stetig.
            \item \f{\frac{f}{g}} ist stetig \f{\forall x} mit \f{g(x) \neq 0}.
            \item Polynome sind stetig.
            \item \f{\sqrt[n]{x}} ist stetig auf \f{\mathbb{R}^+}.
        \end{itemize}
    \end{multicols}
    \item Libschitz-Stetigkeit:\\
    \f{f} heißt Libschitz-stetig falls \f{\exists L > 0} so dass \f{\forall x,x_o \in X} gilt:\\
    \f{|f(x)-f(x_0)| \leq L \cdot |x - x_0|}
    \item Mächtigkeit: \f{f} ist Libschitz-stetig \f{\Rightarrow f} ist gleichmäßig stetig \f{\Rightarrow f} ist stetig
\end{itemize}

\subsection{Funktionenfolgen und -reihen}
\begin{itemize}
    \item Punktweise Konvergenz:\\
    \f{(f_n)} konvergiert punktweise gegen \f{f}, wenn \f{\forall x \in X.} \f{\forall \varepsilon >0.} \f{\exists n_0 \in \mathbb{N}} so dass \f{\forall n \geq n_0} gilt: \f{\quad |f_n(x) - f(x)|<\varepsilon}.\\
    \f{n_0} ist in Abhängigkeit von \f{\varepsilon} und \f{x} zu wählen.
    \item Gleichmäßige Konvergenz:\\
    \f{(f_n)} konvergiert gleichmäßig gegen \f{f}, wenn \f{\forall \varepsilon >0.} \f{\exists n_0 \in \mathbb{N}} so dass \f{\forall n \geq n_0} \f{\forall x \in X} gilt: \f{\quad |f_n(x) -f(x)|<\varepsilon}.\\
    \f{n_0} ist in Abhängigkeit von \f{\varepsilon} zu wählen.
    \item Konvergiert \f{(f_n)} gleichmäßig gegen \f{f} und \f{f_n} ist stetig \f{\forall n \Longrightarrow f} ist stetig.
    \item Funktionenfolge auf Konvergenz überprüfen:\\
    Sei \f{f_n(x)} gegeben. Zu zeigen: Punktweise und gleichmäßige Konvergenz von \f{f_n(x)}.
    \begin{enumerate}
        \item Punktweise: \f{f_n} konvergiert punktweise, wenn die Grenzfunktion\\
        \f{\lim_{n\to\infty}f_n(x)=f(x)} existiert.
        \item Gleichmäßig: Zeige \f{|f_n(x)-f(x)|<\varepsilon}
    \end{enumerate}
\end{itemize}

\subsection{Nützliche Grenzwerte}
\begin{enumerate}
    \item \f{\lim_{z\to\infty}\frac{z^n}{e^z}=0}
    \item \f{\lim_{z\to0}\frac{e^z-1}{z}=1}
    \item \f{\lim_{x\to\infty}\frac{\log(x)}{\sqrt[n]{x}}=0\quad(\log} konvergiert wesentlich langsamer als andere Funktionen)
\end{enumerate}

\subsection{Trigonometrische Funktionen}
\begin{multicols}{2}
    \begin{itemize}
        \item \f{\sin(x)=\frac{e^{ix}-e^{ix}}{2i}=\sum_{n=0}^{\infty}(-1)^n\frac{x^{2n+1}}{(2n+1)!}}
        \item \f{\cos(x)=\frac{e^{ix}+e^{ix}}{2}=\sum_{n=0}^{\infty}(-1)^n\frac{x^{2n}}{(2n)!}}
        \item \f{\tan(x)=\frac{\sin(x)}{\cos(x)}}
        \item \f{\sinh(x)=\frac{e^x-e^{-x}}{2}=\sum_{n=0}^{\infty}\frac{x^{2n+1}}{(2n+1)!}}
        \item \f{\cosh(x)=\frac{e^x+e^{-x}}{2}=\sum_{n=0}^{\infty}\frac{x^{2n}}{(2n)!}}
        \item \f{\tanh(x)=\frac{\sinh(x)}{\cosh(x)}}
        \item \f{\sin^2(x)+\cos^2(x)=1}
        \item \f{\sin(2x)=2\cos(x)\sin(x)}
        \item \f{\cos^2(x)=\frac{1}{2}(1+\cos(2x))}
        \item \f{\sin^2(x)=\frac{1}{2}(1-\cos(2x))}
        \item[\vspace{\fill}]
    \end{itemize}
\end{multicols}
\begin{itemize}
    \item \f{\cos(\frac{\pi}{2}-x)=\sin(x) \quad | \quad \sin(\frac{\pi}{2}-x)=\cos(x)}
    \item \f{\cos(x+y)=\cos(x)\cos(y)-\sin(x)\sin(y)}
    \item \f{\sin(x+y)=\cos(x)\sin(y)-\sin(x)\cos(y)}
    \item \f{\cos(2x)=\cos^2(x)-\sin^2(x)=2\cos^2(x)-1=1-2\sin^2(x)}
\end{itemize}
    
        % \section{Gleichungen \& Grenzwerte}
        % \lipsum[1]
        \section{Differenzialrechnung}
        \subsection{Allgemein}
\begin{itemize}
    \item Definition: Eine Funktion \f{f: D\to\mathbb{R}} heißt differenzierbar an der Stelle \f{x_0\in D}, falls folgender Grenzwert existiert:\\
    \f{\frac{d}{dx}f(x_0)=\lim_{x\to x_0}\frac{f(x)-f(x_0)}{x-x_0}=\lim_{h\to 0}\frac{f(x_0+h)-f(x_0)}{h}}
    \item \textbf{Mittelwertsatz:} Sei \f{f:\left[a,b\right]\to\mathbb{R}} (mit \f{a<b}) stetig in \f{\left[a,b\right]} und differenzierbar in \f{\left(a,b\right)}. Dann gilt:\\
    \f{\exists\gamma\in\left(a,b\right): f'(\gamma)=\frac{f(b)-f(a)}{b-a}}
    \item \textbf{L'Hospital:} Seien \f{f,g:\left(a,b\right)\to\mathbb{C}} differenzierbar, \f{g(x)\neq0} \f{\forall x\in\left(a,b\right)} und \f{\lim_{x\to b}f(x)=\lim_{x\to b}g(x)=0 \textrm{ oder } \infty}. Dann gilt:\\
    \f{\lim_{x\to b}\frac{f'(x)}{g'(x)}} existiert \f{\Longrightarrow \lim_{x\to b}\frac{f(x)}{g(x)}=\lim_{x\to b}\frac{f'(x)}{g'(x)}}
\end{itemize}


\subsection{Ableitungsregeln}
Seien \f{u,v} reellwertige, hinreichend oft differenzierbare Funktionen und \f{a\in \mathbb{R}} eine beliebige Konstante. Es gelten folgende Regeln:
\begin{enumerate}
    \item Konstanten: \f{(a)'=0}
    \item Konstanter Vorfaktor: \f{(a\cdot u)'=au'}
    \item Summenregel: \f{(u\pm v)'=u'\pm v'}
    \item Produktregel: \f{(u \cdot v)'=u'\cdot v + u \cdot v'}
    \item Quotientenregel: \f{(\frac{u}{v})' = \frac{u'v-uv'}{v^2}}
    \item Kettenregel: \f{(u\circ v)'(x)=u'(v(x))v'(x)}
    \item Logarithmische Ableitung: \f{(\ln(u))'=\frac{u'}{u}}
\end{enumerate}

\subsection{Wichtige Ableitungen}
\begin{itemize}
    \item \f{f(x)=a^x \longrightarrow f'(x)=a^x\cdot\ln(a)\longrightarrow f''(x)=a^x\cdot\ln(a)\cdot\ln(a)}
    \item \f{f(x)=\ln(x)\longrightarrow f'(x)=\frac{1}{x}\longrightarrow f''(x)=-\frac{1}{x^2}}
    \item \f{f(x)=\log_{a}(x)\longrightarrow f'(x)=\frac{1}{x\cdot \ln(a)} \longrightarrow f''(x)=-\frac{1}{x^2\cdot \ln(a)}}
\end{itemize}


\subsection{Kurvendiskussion}
\begin{itemize}
    \item Extremum: \f{f'(x_0)=0}
    \item Minimum: \f{x_0} ist Extremum und \f{f''(x_0)>0}
    \item Maximum: \f{x_0} ist Extremum und \f{f''(x_0)<0}
    \item Wendepunkt: \f{x_0} ist Extremum, \f{f''(x_0)=0} und \f{f'''(x)\neq 0}
\end{itemize}
        % \section{Funktionen \& Konvergenz}
        % \lipsum[1]
        % \section{Mittelwertsatz}
        % \lipsum[1]
        \section{Taylorreihen/-polynome}
        Anstatt eine Funktion lokal durch eine lineare Abbildung darzustellen, können wir sie, falls die Funktion genügend oft differenzierbar ist, durch polynomiale Funktionen annähern.\\
Sei \f{D \subset \mathbb{R}} offen, \f{x_0 \in D, f:D\to\mathbb{R}} n-mal differenzierbar.
\begin{itemize}
    \item \textbf{Entwicklungspunkt:} \f{x_0}
    \item \textbf{n-tes Taylorpolynom:} \f{T_{n,x_0}f(x)=\sum_{k=0}^{n}\frac{f^{(k)}(a)}{k!}(x-a)^k}
    \item \textbf{Taylorreihe im Entwicklungspunkt:} \f{T_{x_0}f(x):=\sum_{k=0}^{\infty}\frac{f^{(k)}(a)}{k!}(x-a)^k}
    \item Es gilt: \f{T_{0,x_0}f(x)=f(x_0)}
    \item \f{f:D\to\mathbb{R} n}-mal diff.bar, \f{x_0\in D, f^{'}(x_0)=0, ..., f^{(n-1)}=0, f^{(n)}:}
    \begin{itemize}
        \item \f{n\in2\mathbb{N}, f^{(n)}<0 \Rightarrow x_0} strikt lokales Maximum
        \item \f{n\in2\mathbb{N}, f^{(n)}>0 \Rightarrow x_0} strikt lokales Minimum
        \item \f{n\in2\mathbb{N}+1, f^{(n)}<0 \Rightarrow x_0} kein lokales Extremum
    \end{itemize}
\end{itemize}
    \end{onehalfspace}

    

\end{document}





% Altklausuren

% Wurzrl Sandwich Majoranten / Kriterien

% Ungleichungen (alles auf eine seite -> funktion -> ableiten), Betragsungleichung

% Reihen die als bruch geschrieben werden können um Summe zu eliminieren

% Konvergenz Folgen

% Zwischen/Mittelwertsatz

% Induktionsbew.

% Rekursive Folgen