\documentclass[ngerman,11pt,a4paper
,pdftex]{article}

\usepackage{amsmath}
\usepackage[utf8]{inputenc}
\usepackage{babel}
\usepackage[OT1]{fontenc}
\usepackage{graphicx}
\usepackage{framed}
\usepackage{blindtext}
\usepackage{setspace}
\usepackage{listings}
\usepackage{color}
\usepackage{hyperref}
\usepackage{lipsum}
\usepackage{cleveref}
\usepackage[printonlyused]{acronym}
\usepackage[center]{caption}
\usepackage{tocbibind}
\usepackage{titlesec}
\usepackage{fancyhdr}
\usepackage{tocloft}
\usepackage{latexsym}              
\usepackage[mathletters]{ucs}   
\usepackage[a4paper,hmargin=1cm,vmargin=1cm,landscape=true]{geometry}
\usepackage{amssymb}
\usepackage{amsthm}
\usepackage{multicol}
\usepackage[most]{tcolorbox}
\renewcommand\cftbeforetoctitleskip{-2cm}        % -> To fit TOC on one page

\title{Merkblatt Lineare Algebra 1 \\[1ex] \large Karlsruher Institut für Technologie \\[1ex] Wintersemester 2023/2024}
\author{Niklas Rodenbüsch}
\date{}

\newcommand{\cf}[1]{\[#1\]}
\newcommand{\f}[1]{$#1$}

\newtheoremstyle{nodot}
  {\topsep}   % Space above
  {\topsep}   % Space below
  {}          % Body font
  {}          % Indent amount
  {\bfseries} % Theorem head font
  {}          % Punctuation after theorem head
  { }         % Space after theorem head
  {}          % Theorem head spec (can be left empty, meaning `normal`)

% Define theorem envirnment
\theoremstyle{nodot}
\newtheorem*{definition}{Definition:}

\theoremstyle{nodot}
\newtheorem*{trick}{Trick:}

\begin{document}


    % \pagestyle{fancy}
    % \fancyhf{}
    % \fancyhead[L]{\nouppercase{\leftmark}}
    % \fancyfoot[C]{\thepage}
    % \renewcommand{\headrulewidth}{0.2pt}

    \pagenumbering{gobble}
    % \maketitle
    % \newpage

    % \begin{onehalfspace}
        % \tableofcontents
        % \newpage

        \begin{multicols*}{3}
          \begin{tcolorbox}[colback=white,bottom=10pt, top=10pt,bottom=10pt, top=10pt]
            \subsection*{Gruppen \f{(G,\ast )}}
            \subsubsection*{Halbgruppe}
            \begin{enumerate}
              \item Abgeschlossenheit bzgl. \f{\ast}
              \item Assoziativität
            \end{enumerate}
            \subsubsection*{Monoid}
            \begin{enumerate}
              \item[3.] Neutrales Element
            \end{enumerate}
            \subsubsection*{Gruppe}
            \begin{enumerate}
              \item[4.] Inverses Element
            \end{enumerate}
            \subsubsection*{Abelsche Gruppe}
            \begin{enumerate}
              \item Kommutativität
            \end{enumerate}

            \subsection*{Untergruppen \f{(H,\circ) \text{ von } G}}
            \begin{enumerate}
              \item \f{e_G\in H}
              \item Abgeschlossenheit bzgl. \f{\circ}
              \item Inverses Element
            \end{enumerate}
          \end{tcolorbox}
          
          \begin{tcolorbox}[colback=white,bottom=10pt, top=10pt]
            \subsection*{Ringe \f{(R,+, \cdot)}}
            \begin{enumerate}
              \item \f{(R,+)} abelsche Gruppe
              \item \f{(R,\cdot)} Monoid
              \item Distributivgesetze
              \item \f{\cdot} kann kommutativ sein
            \end{enumerate}
          \end{tcolorbox}

          \begin{tcolorbox}[colback=white,bottom=10pt, top=10pt]
            \subsection*{Körper \f{(K,+, \cdot)}}
            \begin{enumerate}
              \item \f{(K,+)} abelsche Gruppe
              \item \f{(K\backslash 0_K,\cdot)} abelsche Gruppe
              \item Distributivgesetze
            \end{enumerate}
          \end{tcolorbox}

          \begin{tcolorbox}[colback=white,bottom=10pt, top=10pt]
            \subsection*{K-Vektorraum \f{V}}
            \begin{enumerate}
              \item \f{(K)} ist Körper
              \item \f{(V,+)} abelsche Gruppe
              \item Abgeschlossen bzgl. skalarer Multiplikation\\
              \f{\cdot : K\times V \rightarrow V}
            \end{enumerate}
            
            \subsection*{Untervektorraum \f{U \text{ von } V}}
            \begin{enumerate}
              \item \f{0_V \in U}
              \item Abgeschlossen bzgl. Addition
              \item Abgeschlossen bzgl. skalarer Multiplikation
            \end{enumerate}

            \subsection*{Affine Unterräume \f{R}}
            \begin{itemize}
              \item \f{R = \left\{p+x | x\in U\right\} \quad (p\in\mathbb{R}^n)}
            \end{itemize}
          \end{tcolorbox}

        \begin{tcolorbox}[colback=white,bottom=10pt, top=10pt]
          \subsection*{Homomorphismen \f{\Phi}}
          \begin{itemize}
            \item \f{G, H \text{ Strukturen mit Verknüpfungen } \dotplus}
            \item \f{\Phi:G\rightarrow H}
          \item \f{\forall x,y\in G: \Phi(x\dotplus _G y) = \Phi(x)\dotplus_H\Phi(y)}
          \end{itemize}
        \end{tcolorbox}

        \begin{tcolorbox}[colback=white,bottom=10pt, top=10pt,breakable,enhanced]
          \subsection*{Algebra \f{(A,+,\circ,\cdot)}}
          \begin{itemize}
            \item \f{(A,+,\cdot)} K-Vektorraum
            \item \f{(A,+,\circ)} Ring
            \item \f{(\lambda\cdot a)\circ b = a\circ (\lambda \cdot b) = \lambda \cdot (a\circ b)}
          \end{itemize}
        \end{tcolorbox}

        \begin{tcolorbox}[colback=white]
          \subsection*{Matrizen \f{A}}
          \begin{itemize}
            \item \f{A\in\mathbb{R}^{m\times n}\rightarrow} m Zeilen
            \item \f{(AB)^T=B^TA^T,\quad (A+B)^T=A^T+B^T}
          \end{itemize}

          \subsubsection*{Determinante}
          \begin{itemize}
            \item det\f{(\lambda A)=\lambda^m\cdot \text{ det}(A)}
            \item 2\f{\times}2: \f{ab-cd}
            \item 3\f{\times}3: Regel von Sarrus
          \end{itemize}
          \begin{enumerate}
            \item Mit Gauß vereinfachen (Dreiecksmatrix/Zeile mit vielen Nullen)
            \item Nach Zeile/Spalte entwickeln
          \end{enumerate}

          \subsubsection*{Invertierbare Matrizen}
          \begin{itemize}
            \item det\f{(A)\neq0}
            \item ker\f{(A)=\left\{0\right\}} 
            \item rg\f{(A)=m}
            \item \textbf{Vorgehen: } \f{(A|I)\leadsto (I|X)}
            \item \textbf{Cramersche Regel: }\f{A^{-1}=\frac{\text{adj}(A)}{\text{det}(A)}}
            \item \f{A = \begin{pmatrix}a&b\\c&d\end{pmatrix} \rightarrow A^{-1}=\frac{1}{\text{det}(A)}\begin{pmatrix}d&-b\\-c&a\end{pmatrix}}
          \end{itemize}
        \end{tcolorbox}
        
        \begin{tcolorbox}[colback=white,bottom=10pt, top=10pt]
          \subsubsection*{Bild einer Matrix}
          \begin{itemize}
            \item Die linear unabhängigen Spalten
          \end{itemize}
          \subsubsection*{Spur einer Matrix}
          \begin{itemize}
            \item Summe der Diagonalelemente
          \end{itemize}
          \subsubsection*{Rang einer Matrix}
          \begin{itemize}
            \item Anzahl nicht-Null-Zeilen in ZSF
          \end{itemize}
          \subsubsection*{Kern einer Matrix}
          \begin{itemize}
            \item ker\f{(A)=\left\{x\in K^n|Ax=0\right\} }
          \end{itemize}
          \subsubsection*{Dimensionsformel}
          \begin{itemize}
            \item rang\f{(A)+}dim(ker\f{(A)})\f{=n}
          \end{itemize}
          \subsubsection*{Ähnlichkeit von Matrizen}
          \begin{itemize}
            \item Determinante gleich, Spur gleich UND\\
            \f{\exists S} mit \f{B=SAS^{-1}}
          \end{itemize}
          \subsubsection*{Äquivalenz von Matrizen}
          \begin{itemize}
            \item \f{rg(A)=rg(B)}
          \end{itemize}
        \end{tcolorbox}

        \begin{tcolorbox}[colback=white,bottom=10pt, top=10pt]
          \subsection*{Lineare Abbildungen \f{\Phi}}
          \begin{itemize}
          \item \f{V, W} Vektorräume, \f{\Phi:V \rightarrow W}
          \item \f{\forall x,y\in V: \Phi(x+y)=\Phi(x)+\Phi(y)}
          \item \f{\forall x\in V: \Phi(\lambda x)=\lambda \Phi(x)}
          \item \f{\Phi} ist Vektorraum
          \item ker\f{(\Phi) = \left\{v\in V |\Phi(v)=0\right\} }
          \end{itemize}
        \end{tcolorbox}

        \begin{tcolorbox}[colback=white,bottom=10pt, top=10pt]
          \subsection*{Linearkombination}
          Sei V ein K-VR und U ein UVR von V. Seien \f{n\in \mathbb{N}, v_1,...,v_n\in U, \lambda_1,...,\lambda_n\in K.} Dann ist \f{\sum_{i=1}^{n}\lambda_iv_i} eine Linearkombination.\\[10pt]

          \subsection*{Lineare Unabhängigkeit}
          Sei V ein K-VR und M \f{= \left\{m_1,...,m_n\right\}\subseteq V}. M ist linear unabhängig, wenn gilt:\\
          \f{\sum_{i=1}^{n}\lambda_im_i=0\Leftrightarrow \lambda_i=0\quad (\lambda_i\in K)}\\[10pt]

          \subsection*{Lineare Hülle}
          Sei V ein K-VR und M ein UVR von V. LH(M) ist die Menge aller Linearkombinationen aus M. LH(M) ist der kleinste UVR, der M enthält.
          M heißt Erzeugendensystem von LH(M).\\[10pt]
          \subsection*{Basis}
          Sei V ein K-VR und U ein UVR von V. U ist eine Basis von V, wenn gilt:
          \begin{itemize}
            \item \f{V = \text{LH}(U)}
            \item Jedes \f{u\in U} ist linear unabhängig
          \end{itemize}
          \subsection*{Dimension/Rang von \f{\Phi\in \text{Hom}(V,W)}}
          \begin{itemize}
            \item rg\f{(\Phi) = \text{dim(img(}\Phi))}
            \item dim\f{(V)=rg(\Phi)+dim(ker(\Phi))}
          \end{itemize}
          \subsection*{Lineare Fortsetzung}
          Seien V und W K-Vektorräume, B eine Basis von V und \f{I = \left\{1,...,n\right\}}. Dann gibt es genau eine lineare Abbildung \f{\Phi: V\rightarrow W} mit \f{\Phi(b_i)=w_i} für alle \f{i\in I}. 
        \end{tcolorbox}

        \begin{tcolorbox}[colback=white,bottom=10pt, top=10pt]
          \subsection*{Abbildungsmatrix}
          \begin{itemize}
            \item Lineare Abbildung \f{f:\mathbb{R}^n\rightarrow\mathbb{R}^m}
            \item AM spaltenweise: \f{A=(f(e_1)...f(e_n))}\\
            \f{\Rightarrow f(v)=A\cdot v\quad (\forall v \in V)}
          \end{itemize}
          \subsection*{Basiswechsel}
          Sei V ein K-VR mit geordneter Basen B und C. Basiswechsel von B nach C \f{((v)_B\leadsto(v)_C)}:
          \f{(v)_C=M_{CB}(id_V)\cdot(v)_B}
          \subsubsection*{Bestimmung der Basiswechselmatrix}
          \f{M_{CB}(id)=M_{CE}(id)\cdot M_{EB}(id)=M_{EC}(id)^{-1}\cdot M_{EB}(id)} mit \f{M_{EX}(id)=(x_1|...|x_n)}
        \end{tcolorbox}

        \begin{tcolorbox}[colback=white,bottom=10pt, top=10pt]
          \subsection*{Eigenwerte einer Matrix A}
          det\f{(A-\lambda I_n)=0}\\[5pt]
          \subsection*{Charakteristisches Polynom}
          CP(A)\f{=\text{det}(A-\lambda I)}\\[5pt]
          \subsection*{Eigenvektoren}
          x ist ein EV zum Eigenwert \f{\lambda_i}, wenn er folgendes LGS löst:
          \f{(A-\lambda_iI)x=0}\\[5pt]
          \subsection*{Eigenräume zu EW \f{\lambda}}
          Eig(\f{A, \lambda})\f{=\text{ker}(A-\lambda I)}\\[5pt]
          \subsection*{Diagonalisierbarkeit}
          Häufigkeit \f{\lambda = \text{dim(Eig(}A,\lambda))}
        \end{tcolorbox}

        % \begin{tcolorbox}[colback=white,bottom=10pt, top=10pt]{
        %   \begin{itemize}
        %     \item Basiswechsel/Abbildungsmatrizen
        %     \item Eigenwertproblem
        %     \item (Kern, Basis, etc eines Hom)
        %   \end{itemize}}
        % \end{tcolorbox}

        \end{multicols*}
    % \end{onehalfspace}

    

\end{document}