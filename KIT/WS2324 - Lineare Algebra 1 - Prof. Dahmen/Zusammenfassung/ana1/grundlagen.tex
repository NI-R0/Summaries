\subsection{Rechenregeln}
\subsubsection*{Binomische Formeln}
\begin{multicols}{3}
    \begin{itemize}
        \item \f{(a+b)^2=a^2+2ab+b^2}
        \item \f{(a-b)^2=a^2-2ab+b^2}
        \item \f{(a+b)(a-b)=a^2-b^2}
    \end{itemize}
\end{multicols}
\vspace*{1em}
\begin{itemize}
    \item Allg. binomische Formel: \f{(x+y)^n=\sum_{k=0}^{n}{n\choose k}x^{n-k}y^k}
    \item \f{{n\choose k}=\frac{n!}{k!(n-k)!}={n-1\choose k-1}+{n-1\choose k}}, \f{n\ge k > 0}
\end{itemize}

\subsubsection*{Potenzen}
\begin{multicols}{3}
    \begin{itemize}
        \item \f{x^{-n}=\frac{1}{x^n}}
        \item \f{x^a*x^b=x^{a+b}}
        \item \f{\frac{x^a}{x^b}=x^{a-b}}
        \item \f{x^n*y^n=(xy)^n}
        \item \f{x^{-n}=\frac{1}{x^n}}
        \item \f{x^{\frac{1}{n}}=\sqrt[n]{x}}
    \end{itemize}
\end{multicols}

\subsubsection*{Logarithmus}
\begin{multicols}{2}
    \begin{itemize}
        \item \f{\log_{a}(b)=\frac{\ln(b)}{\ln(a)}}
        \item \f{\log_{a}(1)=0}
        \item \f{\log_{a}(b^n)=n*\log_{a}(b)}
        \item \f{\log_{a}(x*y)=\log_{a}(x)+\log_{a}(y)}
        \item \f{\log_{a}(\frac{x}{y})=\log_{a}(x)-\log_{a}(y)}
        \item \f{\log(x+y)=\log(x)+\log(1+\frac{y}{x})}
        %\item[\vspace{\fill}]
    \end{itemize}
\end{multicols}

\subsubsection*{e-Funktion und ln}
\begin{multicols}{3}
    \begin{itemize}
        %\item \f{e = \sum_{n=0}^{\infty} \frac{1}{n!} \approx 2,718 }
        \item \f{\ln(e)=1}, \f{\ln(0)=1}
        \item \f{e^x=b \Leftrightarrow x=\ln(b)}
        \item \f{e^{\ln(a)}=a=\ln(e^a)}
    \end{itemize}
\end{multicols}

\subsection{Summen-/Produktformeln}
\begin{itemize}
    \item Gaußsche Summenformel: \f{\sum_{k=1}^{n}k=\frac{n \cdot(n+1)}{2}}
    \item Teleskopsumme: \f{\sum_{n=0}^{N}(a_{n+1}-a_n)=-a_0+a_{N+1}}
    \item Geometrische Summenformel: \f{\sum_{k=0}^{n}q^k=\frac{1-q^{k+1}}{1-q}}
    \item Teleskopprodukt: \f{\prod_{i=1}^{n}\frac{a_{i+1}}{a_i}=\frac{a_{n+1}}{a_1}}
\end{itemize}


\subsection{Ungleichungen}
\subsubsection{Rechenregeln}
\begin{itemize}
    \item Keine Zeichenänderung bei Addition, Subtraktion oder Multiplikation mit positiver Zahl. (\f{a \leq b} und \f{c \leq d \Rightarrow a+c \leq b+d}),(\f{a\leq b} und \f{c \geq 0 \Rightarrow ac \leq bc})
    \item Zeichenwechsel bei Division und Multiplikation mit negativer Zahl.
    \item Ungleichungen mit Betrag: Fallunterscheidung (\f{\geq 0} bzw \f{< 0})
    \item \f{a \leq b \Rightarrow \frac{1}{a}\geq\frac{1}{b}\quad} (Zeichenwechsel bei Kehrwertbildung)
\end{itemize}
\subsubsection{Wichtige Ungleichungen}
\begin{itemize}
    \item \textbf{Ungleichung vom arithmetischen und geometrischen Mittel:}\\
    Für \f{a_1,a_2,...,a_n\geq 0} gilt: \f{\frac{a_1+a_2+...+a_n}{n}\geq \sqrt[n]{a_1\cdot a_2 \cdot ... \cdot a_n}}
    \item \textbf{Bernoulli-Ungleichung:}\\
    Sei \f{x \in \mathbb{R}} und \f{x \geq -1} und \f{n \in \mathbb{N}}. Es gilt: \f{(1+x)^n \geq 1 +nx}
    \item \textbf{Young'sche Ungleichung:}\\
    Sind \f{p,q>1 \textrm{ mit } \frac{1}{p}+\frac{1}{q}=1 \textrm{ und } a,b\geq0}, dann gilt: \f{ab \leq \frac{a^p}{q}+\frac{b^q}{q}}
\end{itemize}



\subsection{Tricks}
\begin{itemize}
    %\item \f{e = \sum_{n=0}^{\infty} \frac{1}{n!} \approx 2,718 }
    \item Summen auseinanderziehen: \f{\sum_{k=1}^{n+1}k^3=\sum_{k=1}^{n}k^3+(n+1)^3}
    \item Dritte bin. Formel für Wurzelfolgen: \f{a-b=\frac{a^2-b^2}{a+b}}\\
    Beispiel: \f{\sqrt{x}-\sqrt{y}=\frac{x-y}{\sqrt{x}+\sqrt{y}}}
\end{itemize}

% \subsection{Polynomdivision}


%TODO: Polarkoordinaten (komplexe Zahlen), Tricks zur Grenzwertbestimmung, Taylor 
