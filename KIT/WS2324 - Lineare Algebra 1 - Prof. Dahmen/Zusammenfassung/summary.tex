\documentclass[ngerman,12pt,a4paper
,pdftex]{article}

\usepackage{amsmath}
\usepackage[utf8]{inputenc}
\usepackage{babel}
\usepackage[OT1]{fontenc}
\usepackage{graphicx}
\usepackage{framed}
\usepackage{blindtext}
\usepackage{setspace}
\usepackage{listings}
\usepackage{color}
\usepackage{hyperref}
\usepackage{lipsum}
\usepackage{cleveref}
\usepackage[printonlyused]{acronym}
\usepackage[center]{caption}
\usepackage{tocbibind}
\usepackage{titlesec}
\usepackage{fancyhdr}
\usepackage{tocloft}
\usepackage{latexsym}              
\usepackage[mathletters]{ucs}   
\usepackage[a4paper,hmargin=2.5cm,bottom=3cm]{geometry}
\usepackage{amssymb}
\usepackage{amsthm}
\usepackage{multicol}
\renewcommand\cftbeforetoctitleskip{-2cm}        % -> To fit TOC on one page

\title{Zusammenfassung Lineare Algebra 1 \\[1ex] \large Karlsruher Institut für Technologie \\[1ex] Wintersemester 2023/2024}
\author{Niklas Rodenbüsch}
\date{}

\newcommand{\cf}[1]{\[#1\]}
\newcommand{\f}[1]{$#1$}

\newtheoremstyle{nodot}
  {\topsep}   % Space above
  {\topsep}   % Space below
  {}          % Body font
  {}          % Indent amount
  {\bfseries} % Theorem head font
  {}          % Punctuation after theorem head
  { }         % Space after theorem head
  {}          % Theorem head spec (can be left empty, meaning `normal`)

% Define theorem envirnment
\theoremstyle{nodot}
\newtheorem*{definition}{Definition:}

\theoremstyle{nodot}
\newtheorem*{trick}{Trick:}

\begin{document}


    \pagestyle{fancy}
    \fancyhf{}
    \fancyhead[L]{\nouppercase{\leftmark}}
    \fancyfoot[C]{\thepage}
    \renewcommand{\headrulewidth}{0.2pt}

    \maketitle
    \newpage
    \pagenumbering{Roman}
    

    \begin{onehalfspace}
        % \tableofcontents
        % \newpage
        \newcounter{savepage}
        \setcounter{savepage}{\value{page}}
        \pagenumbering{arabic}

        \section{Grundlagen}
        % \subsection*{Allgemeines}
% \begin{itemize}
%     \item \f{sin(x\pm y) = sin(x)\cdot cos(y) \pm cos(x)\cdot sin(y)}\\
%     \f{cos(c\pm y) = cos(x)\cdot cos(y) \mp sin(x)\cdot sin(y)}
% \end{itemize}

\subsection*{Abbildungen}
Sei \f{f:A \rightarrow B}. Es gilt:
\begin{itemize}
    \item f ist \textbf{injektiv}, wenn: \f{\forall a_1, a_2 \in A : f(a_1) = f(a_2) \Rightarrow a_1 = a_2}
    \item f ist \textbf{surjektiv}, wenn: \f{\forall b \in B\quad  \exists a \in A : f(a) = b}
    \item f ist \textbf{bijektiv}, wenn f injektiv und surjektiv ist
\end{itemize}

\subsection*{Relationen}
Sei \f{x,y,z \in M} und \f{R} eine Relation:
\begin{itemize}
    \item \textbf{Reflexiv}, wenn \f{xRx}
    \item \textbf{Symmetrisch}, wenn \f{xRy \Leftrightarrow yRx}
    \item \textbf{Antisymmetrisch}, wenn \f{xRy \wedge yRz \Rightarrow x=y}
    \item \textbf{Transitiv}, wenn \f{xRy \wedge yRz \Rightarrow xRz}
    \item \textbf{Äquivalent}, wenn \f{R} reflexiv, symmetrisch und transitiv ist.\\
\end{itemize}
Sei \f{\sim } eine Äquivalenzrelation auf \f{M}.
\begin{itemize}
    \item \f{\left[x\right] _{\sim} = \left\{y\in M | x \sim y\right\}} ist \textbf{Äquivalenzklasse}
    \item \f{M/ \sim = \left\{\left[x\right] _{\sim} | x \in M\right\} }
    \item \f{\mathbb{Z} /n\mathbb{Z} = \left\{\left[0\right],\left[1\right], ...,\left[n-1\right] \right\}}
\end{itemize}
        \section{Lineare Gleichungssysteme}
        \begin{itemize}
    \item Lösung eines LGS: Vektor \f{x = \begin{pmatrix}x_1 \\ \vdots  \\ x_n \end{pmatrix}\in \mathbb{R}^{n}}, der gleichzeitig alle Gleichungen erfüllt.
    \item Unendlich viele Lösungen, wenn bei Gauß-Umformung eine Nullzeile entsteht.
    \item LGS heißt homogen, wenn \f{x_1 = ... = x_n = 0}
\end{itemize}
        \section{Matrizen}
        \subsection{Allgemein}
\begin{itemize}
    \item Sei \f{A\in \mathbb{R}^{m\times n}} eine \f{(m\times n)}-Matrix. Dann hat \f{A} \f{m} Zeilen und \f{n} Spalten.
    \item \textbf{Quadratisch}, wenn \f{m=n}
    \item Transponierte zu \f{A: \quad A^{T}\in\mathbb{R}^{n\times m}} mit \f{(A^T)_{i,j}=A_{j,i}}
    \item Sei \f{A\in\mathbb{R}^{m\times k}, B\in\mathbb{R}^{k\times n}}. Dann ist \f{C:=A\cdot B \in \mathbb{R}^{m\times n}} mit \f{C_{i,j}=\sum_{t=1}^{k}A_{i,t}\cdot B_{t,j}}\\
    (A wird zeilenweise und B spaltenweise durchlaufen)
    \item Ein LGS kann als Matrix-Vektor-Produkt geschrieben werden.
\end{itemize}
\subsection{Rechenregeln}
\begin{multicols}{2}
    \begin{itemize}
        \item \f{(\lambda A)^T = \lambda A^T}
        \item \f{(A +B)^T=A^T+B^T}
        \item \f{(AB)^T=B^TA^T}
        \item \f{(A^T)^T=A}
    \end{itemize}
\end{multicols}

\subsection{Invertierbare Matrizen}
\begin{itemize}
    \item Sei R ein Ring. Eine Matrix \f{A\in R^{m\times m} } heißt invertierbar, falls\\ 
    \f{\exists B \in R^{m\times m}: AB = BA = I_m}
    \item GL\f{_m(R)} ist die Menge der invertierbaren Matrizen aus \f{R^{m\times m}} 
    \item Wenn \f{A\in R^{m\times m}} invertierbar ist, gilt:\\
    \f{\Leftrightarrow A^T} ist invertierbar\\
    \f{\Leftrightarrow} ker\f{(A) = \left\{0\right\}}\\
    \f{\Leftrightarrow} rg\f{(A) = m}
    \item \textbf{Vorgehen:} Mittels Gauß-Algorithmus \f{(A|I)\leadsto (I|X)\Longrightarrow X = A^{-1}}
\end{itemize}

\subsection{Bild einer Matrix}
Das Bild einer Matrix gibt an, welche Menge an Vektoren als Lösungen auftreten können.\\
Das Bild einer linearen Abbildung \f{f:V\rightarrow W} ist die Menge aller Vektoren in \f{W}, die von \f{f} getroffen werden.

\subsection{Zeilenstufenform}
Wenn eine Matrix die Form \f{\begin{pmatrix}* & * & * & * \\ 0 & * & * & * \\ 0 & 0 & 0 & *\\ 0&0&0&0\end{pmatrix}} hat, ist sie in Zeilenstufenform. Wenn sie die Form \f{\begin{pmatrix}1 & 0 & * & 0 \\ 0 & 1 & * & 0 \\ 0 & 0 & 0 & 1\\ 0&0&0&0\end{pmatrix}} hat, ist sie in normierter Zeilenstufenform. 

\subsubsection*{Eigenschaften der Zeilenstufenform}
\begin{itemize}
    \item Der \textbf{Rang} der Matrix ist die Anzahl der Stufen.
    \item Die Anzahl der Lösungen lässt sich wie folgt ablesen:
    \begin{itemize}
        \item \textbf{Keine Lösung:} Wenn eine Zeile die Form \f{(0 ... 0|b\neq 0)} hat.
        \item \textbf{Eine Lösung:} Wenn die Matrix in optimaler Form ist.
        \item \textbf{Mehrere Lösungen:} Wenn die Matrix eine Zeile \f{(0...0|0)} hat.
    \end{itemize}
\end{itemize}

\subsection{Ähnlichkeit von Matrizen}
Seien \f{A,B\in K^{n\times n}} quadratisch. \f{A} und \f{B} sind ähnlich wenn:
\begin{center}
    \f{\exists S\in GL_n(K): \quad B = SAS^{-1}}
\end{center}

\subsection{Äquivalenz von Matrizen}
Seien \f{A,B\in K^{m\times n}}. \f{A} und \f{B} sind äquivalent, wenn:
\begin{center}
    \f{\exists T \in GL_m(K), S\in GL_n(K): \quad B = TAS}\\
    \f{A} und \f{B} äquivalent \f{\Leftrightarrow} Rang(\f{A}=) Rang(\f{B})
\end{center}

\subsection{Spur einer Matrix}
Sei \f{A\in K^{n\times n}} eine quadratische Matrix. Die Spur ist definiert als:
\begin{center}
    tr\f{(A)=\text{Spur}(A)=\sum_{i=1}^{n}a_{ii}}
\end{center}



        \section{Algebraische Strukturen}
        Es sei \f{M} eine Menge und \f{a,b \in M}. Eine Verknüpfung ist eine Abbildung \f{*: M\times M \rightarrow M} mit folgenden möglichen Eigenschaften:
\begin{itemize}
    \item Assoziativität: \f{(x*y)*z=x*(y*z)}
    \item Kommutativität: \f{x*y=y*x}
\end{itemize}



\subsection{Gruppen}
\subsubsection*{Halbgruppen}
Sei \f{S} eine Menge und \f{*} eine Verknüpfung. Eine Halbgruppe \f{S,*} erfüllen folgende Eigenschaften:
\begin{enumerate}
    \item \f{*:S\times S \rightarrow S \quad (\forall a,b \in S: a*b \in S)\quad} (\textbf{Abgeschlossenheit})
    \item \f{*} ist \textbf{assoziativ}
\end{enumerate}
\subsubsection*{Monoid}
Eine Halbgruppe ist insbesondere ein Monoid, wenn zusätzlich gilt:
\begin{enumerate}
    \item[3.] \f{\exists e \in S:\forall x\in S: x*e = e*x = x\quad} (\textbf{Existenz eines neutralen Elements})
\end{enumerate}
\subsubsection*{Gruppen}
Ein Monoid ist insbesondere eine Gruppe, wenn zusätzlich gilt:
\begin{enumerate}
    \item[4.] \f{\forall x \in S: \exists y =: x^{-1}\in S: x*y=y*x=e\quad} (\textbf{Existenz des inversen Elements}) 
\end{enumerate}
\subsubsection*{Abelsche Gruppe}
Sei \f{(G,*)} eine Gruppe. Damit \f{(G,*)} eine abelsche Gruppe ist, muss zusätzlich gelten:
\begin{enumerate}
    \item[5.] * ist \textbf{kommutativ} 
\end{enumerate}
\subsubsection*{Symmetrische Gruppe}
\begin{itemize}
    \item Symmetrische Gruppe \f{\mathcal{S}(X):=(X^X, \circ )^\times = (\left\{f:X\rightarrow X |f\text{ bijektiv}\right\} , \circ)}
    \item Für \f{X = \left\{1,...,n\right\} : \mathcal{S}(n):=\mathcal{S}(X)\quad(n\in\mathbb{N})}
    \item Permutationen \f{\sigma \in\mathcal{S}(n): \sigma = \begin{pmatrix}1&2&...&n\\\sigma(1)&\sigma(2)&...&\sigma(n)\end{pmatrix}}
\end{itemize}
\newpage
\subsubsection*{Untergruppen}
Sei \f{(G,*)} eine Gruppe. Eine Gruppe \f{(H,*)} heißt \textbf{Untergruppe} von \f{(G,*)} wenn folgende Eigenschaften gelten:
\begin{enumerate}
    \item \f{e_G \in H}
    \item \f{\forall g,h \in H: g*h \in H}
    \item \f{\forall g \in H: g^{-1} \in H}
\end{enumerate}
\subsubsection*{Homomorphismen}
Homomorphismen sind strukturerhaltende Abbildungen. Seien \f{(G,*),(H,\bullet)} Gruppen. Dann ist \f{f:G\rightarrow H \in\text{ Hom}(G,H)} ein Gruppenhomomorphismus, wenn gilt:\\
\f{\forall x,y\in G: f(x*y)=f(x)\bullet f(y)}\\

\noindent{}Sei \f{h\in\text{Hom}(G,H)}, dann gelten folgende Eigenschaften:
\begin{itemize}
    \item \f{h(e_G)=e_H}
    \item \f{h(g^{-1}=h(g)^{-1})}
    \item \f{U \text{ ist UGR von } G \Rightarrow h(U) \text{ ist UGR von } H}
    \item \f{h} injektiv \f{\Leftrightarrow} Kern \f{h := h^{-1}(\left\{e_H\right\} )=\left\{e_G\right\} }\\
\end{itemize}

\noindent{}Sei \f{h\in \text{Hom}(G,H)}. Dann ist Kern\f{(h)} definiert als:\\
\f{\text{Kern }h:=h^{-1}(\left\{e_H\right\}) = \left\{g\in G: h(g)=e_H\right\}}

\subsection{Ringe}
Ein Ring \f{(R,+,\cdot)} erfüllt die folgenden Eigenschaften:
\begin{enumerate}
	\item \f{(R,+)} ist eine abelsche Gruppe (mit neutralem Element \f{0_R})
	\item \f{(R,\cdot)} ist Monoid (mit neutralem Element \f{1_R})
	\item Für alle \f{x,y,z \in R} gelten die Distributivgesetze
	\item \f{R} kommutativ \f{:\Leftrightarrow \cdot} kommutativ
	\item Nullteilerfrei, falls: \f{\forall a,b\in R: (a\cdot b = 0_R)\Rightarrow(a=0_R \vee b=0_R)}
\end{enumerate}
\subsubsection*{Ringhomomorphismus}
Seien \f{(R, +_R, \cdot_R), (S, +_S, \cdot_S)} Ringe. Ein Ringhomomorphismus \f{\Phi : R\rightarrow S} erfüllt folgende Eigenschaften (\f{\forall x,y \in R}):
\begin{itemize}
    \item \f{\Phi (x+_Ry)=\Phi(x) +_S \Phi(y)}
    \item \f{\Phi(x\cdot_Ry)=\Phi(x)\cdot_S\Phi(y)}
    \item \f{\Phi(1_R)=1_S}
    \item Kern \f{\Phi = \Phi^{-1}(\left\{0_S\right\})}
\end{itemize}

\subsection{Körper}
Ein Körper \f{(K,+,\cdot)} erfüllt die folgenden Eigenschaften:
\begin{enumerate}
    \item \f{(K,+)} ist eine abelsche Gruppe
    \item \f{(K\backslash 0_K), \cdot} ist eine abelsche Gruppe
    \item Für alle \f{x,y,z \in K} gilt das Distributivgesetze\\
\end{enumerate} 

\noindent{}\textbf{Restklassenkörper:} \f{\mathbb{F}_p := \mathbb{Z} /p\mathbb{Z}}, \f{p \in \mathbb{N}} prim
\subsubsection*{Körperhomomorphismus}
Seien \f{(K,+,\cdot)} und \f{(L,+,\cdot)} Körper. \f{\Phi:K\rightarrow L} ist ein Körperhomomorphismus, wenn:
\begin{itemize}
    \item \f{\Phi (x+y)=\Phi(x)+\Phi(y)} und \f{\Phi(x\cdot y)=\Phi(x)\cdot\Phi(y)\quad(\forall x,y\in K)}
    \item \f{\Phi(1_K)=1_L}
\end{itemize}

\subsection{Polynomring}
Sei \f{R} ein kommutativer Ring. Dann ist \f{p = a_nX^n + a_{n-1}X^{n-1}+...+a_1X+a_0 \in R\left[X\right] } mit \f{a_i \in R} und folgenden Eigenschaften:
\begin{itemize}
    \item Grad\f{(p)=n\quad} (\f{n=0 \Rightarrow \text{Grad}(p)=-\infty}) 
    \item \f{R\left[X\right] } ist ein Ring (bzw. \f{R}-Algebra falls \f{R} ein Körper ist)
    \item \textbf{Einsetzabbildung:} \f{f \in R\left[X\right] \mapsto f:R\rightarrow R\quad}\\
    (Ersetzen von \f{X} durch ein Element aus \f{R}) 
\end{itemize}
\newpage
        \section{Vektorräume und lineare Abbildungen}
        \subsection{Vektorräume}
\f{V} ist ein \f{K}-Vektorraum, wenn folgende Eigenschaften erfüllt sind:
\begin{itemize}
    \item \f{K} ist ein Körper
    \item \f{(V,+)} ist eine abelsche Gruppe
    \item \f{\cdot : K \times V \rightarrow V\quad} (Skalarmultiplikation)
\end{itemize}
Mit \f{u,v \in V \text{ und } \lambda, \mu \in K} gelten folgende Rechenregeln:
\begin{itemize}
    \item \f{1_K\cdot v = v}
    \item \f{\lambda \cdot (\mu \cdot v) = (\lambda\cdot\mu)\cdot v}
    \item \f{\lambda\cdot(u+v)=\lambda\cdot u + \lambda\cdot v}
    \item \f{(\lambda + \mu)\cdot v = \lambda \cdot v + \mu \cdot v}
\end{itemize}

\subsection*{Untervektorraum}
Sei \f{V} ein \f{K}-VR und \f{U\subseteq V} ein Untervektorraum von \f{V}. Dann ist \f{U} ein \f{K}-VR mit der selben Verknüpfung und Skalarmultiplikation wie \f{V}. \f{U} muss zusätzlich folgende Kriterien erfüllen:
\begin{enumerate}
    \item \f{0 \in U}
    \item \f{\forall v,w \in U: v+w\in U}
    \item \f{\forall v \in U, \forall\lambda \in K: \lambda v \in U}
\end{enumerate}
\subsubsection*{Kombination von UVR}
Sei \f{V} ein \f{K}-Vektorraum und \f{U,W \subseteq  V}.
\begin{itemize}
    \item \f{U\cap W} ist ein \f{K}K-VR
    \item \f{U \cup W} ist i.d.R. \textbf{kein} \f{K}-VR
    \item \f{U+W := \text{LH}(U\cup W) = \left\{u+w|u\in U, w \in W\right\} } ist ein \f{K}-VR
\end{itemize}


%\subsection{}

\subsection{Lineare Abbildungen}
Vektorraumhomomorphismen werden auch als lineare Abbildungen bezeichnet. Seien \f{V,W} \f{K}-Vektorräume, dann ist \f{\Phi : V \rightarrow W \in \text{Hom}_K(V,W)} eine \f{K}-lineare Abbildung, wenn folgende Eigenschaften gelten:
\begin{enumerate}
    \item \f{\forall x,y\in V: \Phi(x+y)=\Phi(x)+\Phi(y)}
    \item \f{\forall x\in V, \lambda \in K: \Phi(\lambda x)=\lambda\Phi(x)}
\end{enumerate}
Des weiteren gilt:
\begin{itemize}
    \item Hom\f{_K(V,W)} ist selbst ein Vektorraum
    \item Kern\f{(\Phi) := \left\{v\in V | \Phi(v)=0\right\}}
    \f{\Phi} injektiv \f{\Longrightarrow \text{Kern}(\Phi) = \left\{0\right\} }
\end{itemize}
\subsection{Lineare Hülle, Basis, Dimension, ...}
\subsubsection*{Linearkombination}
Sei \f{V} ein \f{K}-Vektorraum und \f{M} ein UVR von \f{V}. Seien \f{n\in\mathbb{N}, v_1,...,v_n \in M, \lambda_1,...,\lambda_n \in K}. Dann ist \f{\sum_{i=1}^{n}\lambda_i\cdot v_i \in V} eine Linearkombination.
\subsubsection*{Lineare Hülle}
Sei \f{V} ein \f{K}-Vektorraum und \f{M} ein UVR von \f{V}. Die lineare Hülle ist die Menge aller Linearkombinationen aus M. Die lineare Hülle ist definiert als:\\
\f{\text{LH}(M):=\left\{\sum_{i=1}^{n}\lambda_i\cdot v_i | n \in\mathbb{N}_0, v_i\in M, \lambda_i \in K\right\} }\\

\noindent{}LH\f{(M)} ist der kleinste Untervektorraum, der \f{M} enthält.\\
\f{M} heißt Erzeugendensystem von LH\f{(M)}.\\
\f{K^p = \text{LH}(\left\{v_1,...,v_p\right\}) \Leftrightarrow Rang((v_1|...|v_p))=p}
\subsubsection*{Basis}
Sei \f{V} ein \f{K}-Vektorraum. Ein UVR \f{U \subseteq  V} ist eine Basis von \f{V}, wenn folgende Eigenschaften gelten:
\begin{enumerate}
    \item \f{V = LH(U)}
    \item Jedes \f{u\in U} ist linear unabhängig
\end{enumerate}
\subsubsection*{Lineare Unabhängigkeit}
Sei \f{V} ein \f{K}-Vektorraum und \f{M = \left\{m_1,...,m_m\right\} \subseteq  V}. M ist linear unabhängig, wenn gilt:\\
\f{\sum_{i=1}^{n}\lambda_im_i = 0 \Leftrightarrow \forall i : \lambda_i = 0\quad (\lambda_i \in K)}
\subsubsection*{Dimension}
Sei \f{V} ein \f{K}-Vektorraum, \f{U,W \subseteq V} und \f{B} eine Basis von \f{V}. Dann gilt für die Dimension:
\begin{itemize}
    \item dim\f{(V) = |B|}
    \item dim\f{(U)\leq\text{dim}(V)}
    \item dim\f{(U)=\text{dim}(V) \Leftrightarrow U = V}
    \item Die Dimension ist abhängig vom Körper des Vektorraums.
    \item dim\f{(U+W)=\text{dim}(U)+\text{dim}(W)-\text{dim}(U\cap W)}
    \item \f{W} ist Komplement von \f{U \Leftrightarrow \text{dim}(U)+\text{dim}(W)=\text{dim}(V)}
\end{itemize}
\subsubsection*{Isomorphismen von Basen}
Seien \f{V,W K}-Vektorräume und \f{\Phi \in} Hom\f{(V,W)}. Dann gilt:
\begin{itemize}
    \item \f{M\subseteq V} Erzeugendensystem von V, \f{\Phi} surjektiv \f{\Rightarrow \Phi(M)} Erzeugendensystem
    \item \f{L\subseteq V} linear Unabhängigkeit, \f{\Phi} injektiv \f{\Rightarrow \Phi(L)} linear unabhängig
    \item \f{B\subseteq V} Basis von \f{V, \Phi} bijektiv \f{\Rightarrow \Phi(B)} Basis von \f{W}
\end{itemize}

\subsection{Rechentechniken}
\includegraphics*[width=\textwidth]{sections/Screenshot 2024-03-11 132518.png}



\subsection{Faktorraum/Quotientenraum}
Sei \f{V} ein \f{K}-Vektorraum, \f{U} ein UVR von \f{V} und \f{\sim} eine Äquivalenzrelation. Dann ist ein Faktorraum \f{V/U} definiert duch:
\begin{itemize}
    \item \f{V/U := V/\sim \quad} (Menge der Äquivalenzklassen von \f{\sim})
    \item \f{V/U} ist ein \f{K}-VR mit \f{\left[v\right] +\left[w\right] :=\left[v+w\right] , \lambda \left[v\right]  := \left[\lambda v\right] }
\end{itemize}
\subsection{Rang eines Homomorphismus}
Sei \f{\Phi\in} Hom\f{(V,W)}. Dann ist der Rang von \f{\Phi} definiert als:
\begin{center}
    rg\f{(\Phi) :=} dim(Bild(\f{\Phi}))
\end{center} 
Und es gilt:
\begin{center}
    dim\f{V=} rg(\f{\Phi })\f{+} dim(Kern(\f{\Phi}))\f{=}dim(Bild(\f{\Phi}))\f{+}dim(Kern(\f{\Phi}))
\end{center}


\subsection{Lineare Fortsetzung}
Seien \f{V, W K}-Vektorräume und \f{B} eine Basis von \f{V}. Dann ist \f{\Phi\in\text{Hom}(V,W)} eindeutig definiert durch \f{\Phi|_B}.


\subsection{Basiswechsel}
\subsubsection*{Basisdarstellung}
Sei \f{V} ein \f{K}-Vektorraum und \f{\mathsf{B} =(b_1,...,b_n)} eine geordnete Basis von \f{V}.
\begin{itemize}
    \item \f{\sum_{i}^{n}\lambda_ib_i=v\in V\quad} (Basiseinstellung eindeutig)
    \item Isomorphismus \f{(\cdot)_B: V\rightarrow K^n, v\mapsto (\lambda_1,...,\lambda_n)\quad} (auch rückwärts)
\end{itemize}
\subsubsection*{Abbildungsmatrix}
Es gilt:
\begin{center}
    \f{\forall \Phi\in\text{Hom}(K^n,K^m)\exists A\in K^{m\times n}: \quad \Phi(v)=A\cdot v\quad (v\in K^n)}
\end{center}
Erweiterung: Seien \f{V,W} \f{K}-Vektorräume mit \f{B=(b_1,...,b_n)} Basis von \f{V} und \f{C=(c_1,...,c_n)} Basis von \f{W}. Dann gilt:
\begin{center}
    \f{\forall\Phi\in\text{Hom}(V,W) \exists M_{CB}(\Phi)\in K^{m\times n}: (\Phi(v))_C = M_{CB}(\Phi)\cdot (v)_B\quad (v\in V)}
\end{center}
\subsubsection*{Basiswechsel}
Sei \f{V} ein \f{K}-Vektorraum mit Basis \f{\mathsf{B}=(b_1,...,b_n) } und \f{\mathsf{C}=(c_1,...,c_n) }. Basiswechsel von B nach C (\f{(v)_B\leadsto (v)_C}):
\begin{center}
    \f{(v)_C = M_{CB}(id_V)\cdot (v)_B\quad (v\in V)}
\end{center}
Seien \f{V,W,T} \f{K}-Vektorräume mit je \f{\mathsf{B} ,\mathsf{C} ,\mathsf{D} } geordneten Basen. Seien \f{\Phi\in\text{Hom}(V,W)} und \f{\Psi \in\text{Hom}(W,T)}. Es gilt:
\begin{center}
    \f{M_{FB}(\Psi\circ \Phi) = M_{DF}(\Psi)\cdot M_{CB}(\Psi)}
\end{center}
Sei \f{\Phi} bijektiv, dann gilt:
\begin{center}
    \f{M_{BC}(\Phi^{-1})=M_{CB}(\Phi)^-1}
\end{center}
\subsubsection*{Bestimmen der Basiswechselmatrix}
Sei \f{V=K^n} und \f{E} die Standardbasis. Dann gilt:
\begin{center}
    \f{M_{CB}(id)=M_{CE}(id)\cdot M_{EB}(id) = M_{EC}(id)^{-1}\cdot M_{EB}(id)\quad} mit \f{M_{EX}(id)=(x_1|...|x_n)}
\end{center}


\subsection{Affine Räume}
\subsubsection*{Affiner Unterraum}
Affine Unterräume sind verschobene Vektorräume. Sei \f{U} ein UVR und \f{p\in\mathbb{R}^n}:
\begin{center}
    \f{R = p+U := \left\{p+x|x\in U\right\} }
\end{center}

\subsubsection*{Affine Kombinationen}
Seien \f{n\in\mathbb{N}, v_1,...,v_n\in\mathbb{R}^n, \lambda_1,...,\lambda_n\in\mathbb{R}} dann ist \f{\sum_{i=1}^{n}\lambda_iv_i\in V \text{ mit } \sum_{i=1}^{n}\lambda_i=1} eine Affinkombination.
        \section{Endomorphismen}
        \subsection{Algebren}
Sei \f{K} ein Körper. \f{(A,+,\bullet,\cdot)} ist eine Algebra, wenn sie folgende Eigenschaften erfüllt:
\begin{enumerate}
    \item \f{(A,+,\cdot) \text{ ist ein } K}-Vektorraum
    \item \f{(A,+,\bullet)} ist ein Ring
    \item \f{(\lambda \cdot a)\bullet b = a \bullet(\lambda\cdot b)=\lambda\cdot(a\bullet b)\quad (\forall\in K, a,b\in A)}
\end{enumerate}
\subsection{Determinante von Matrizen}
Seien \f{M,N \in K^{n\times n}}. Dann gilt:
\begin{itemize}
    \item det\f{(M^T)=\text{ det}(M)}
    \item \f{M\in\text{GL}_n(K) (M\text{ invertierbar)} \Leftrightarrow \text{det}(M) \neq 0}
    \item det\f{(M\cdot N) =} det\f{M\cdot} det\f{N}
    \item det\f{(\lambda M) = \lambda^n \cdot \text{det}(M)}
    \item det\f{A^{-1}=\frac{1}{\text{det}(A)}}
    \item det\f{\begin{pmatrix}a_1&...&*\\0&\ddots &*\\0&0&a_n\end{pmatrix} = a_1\cdot ... \cdot a_n}
\end{itemize}
\subsubsection*{Determinante berechnen}
\begin{enumerate}
    \item[0.] det\f{\begin{pmatrix}a&b\\c&d\end{pmatrix} = ad-bc}
    \item Solange wie möglich mit Gauß vereinfachen (Ziel: Dreiecksmatrix oder Zeile/Spalte mit vielen Nullen)
    \item Wenn nicht weiter möglich, nach passender Zeile/Spalte entwickeln
\end{enumerate}
\subsubsection*{Adjunkte}
Sei \f{M=\begin{pmatrix}a&b&c\\d&e&f\\g&h&i\end{pmatrix}} eine quadratische Matrix. Die Adjunkte adj(\f{M}) von \f{M} bestimmt man nach dem folgenden Schema:
\begin{center}
    \f{adj(M)=\begin{pmatrix}\text{det}\begin{pmatrix}e&f\\h&i\end{pmatrix}&-\text{det}\begin{pmatrix}d&f\\g&i\end{pmatrix}&\text{det}\begin{pmatrix}d&e\\g&h\end{pmatrix}\\-\text{det}\begin{pmatrix}b&c\\h&i\end{pmatrix}&\text{det}\begin{pmatrix}a&c\\g&i\end{pmatrix}&-\text{det}\begin{pmatrix}a&b\\g&h\end{pmatrix}\\\text{det}\begin{pmatrix}b&c\\e&f\end{pmatrix}&-\text{det}\begin{pmatrix}a&c\\d&f\end{pmatrix}&\text{det}\begin{pmatrix}a&b\\d&e\end{pmatrix}\end{pmatrix}^T}
\end{center}

\subsubsection*{Cramersche Regel}
Die Cramersche Regel kann zur Berechnung von inversen Matrizen verwendet werden.
\cf{A^{-1}=\frac{\text{adj}(A)}{\text{det}(A)}}

\subsection{Endomorphismen}
Seien \f{V} ein \f{K}-Vektorraum, \f{B} und \f{C} eine Basis von \f{V} und \f{\Phi\in\text{Hom}(V,V)}. Dann ist die Determinante des Endomorphismus:
\begin{center}
    det\f{(\Phi):=\text{det }D_{BB}(\Phi) = \text{det }D_{CC}(\Phi)}
\end{center}
Sei zusätzlich U ein Untervektorraum von V. Dann gilt:
\begin{center}
    \f{U} ist \f{\Phi}-invariant \f{\Leftrightarrow \Phi(U)\subseteq U}
\end{center}





\subsection{Eigenwerte, Eigenvektoren und Eigenräume}
Sei \f{\Phi\in\text{ Hom}(V,V)}. Dann ist:
\begin{itemize}
    \item \f{v\in V\backslash\left\{0\right\}} Eigenvektor von \f{\Phi\quad\Leftrightarrow\quad \Phi(v)=\lambda v\quad(\lambda\in K)}
    \item \f{\lambda\in K} heißt Eigenwert von \f{\Phi\quad\Leftrightarrow\quad\exists} Eigenvektor \f{v} mit \f{\Phi(v)=\lambda v}
    \item Eigenraum Eig\f{(\Phi,\lambda)} ist die Menge aller Eigenvektoren zum Eigenwert \f{\lambda} (und \f{0})\\
    Eig\f{(\Phi, \lambda)} = Kern(\f{\Phi-\lambda\cdot\text{Id}_V})
    \item Spektrum Spec(\f{\Phi}) ist die Menge aller Eigenwerte
    \item Eigenvektoren zu unterschiedlichen Eigenwerten sind linear unabhängig
\end{itemize}

\subsection{Charakteristisches Polynom}
Sei \f{\Phi\in} Hom\f{(V,V)} und \f{B} eine Basis von \f{V}. Das charakteristische Polynom wird zum einfachen Bestimmen von Eigenwerten verwendet.
\begin{center}\f{CP_\Phi(X):=} det\f{(XI_n-D_{BB}(\Phi))}\\\end{center}
Unabhängig von der gewählten Basis \f{B} gilt:
\begin{center}\f{CP_\Phi(\lambda)=0 \Leftrightarrow \lambda} Eigenwert von \f{\Phi}\end{center}

\subsection{Diagonalisierbarkeit}
Sei \f{\Phi\in} Hom\f{(V,V)} diagonalisierbar. Dann gilt:
\begin{itemize}
    \item \f{\exists} Basis \f{B} sodass \f{D_{BB}(\Phi)} in Diagonalform (falls dim\f{V<\infty})
    \item \f{V} hat eine Basis aus Eigenvektoren von \f{\Phi} 
    \item \f{V} ist Summe der Eigenräume
    \item Charakteristisches Polynom zerfällt in Linearfaktoren und geometrische und algebraische Vielfachheiten stimmen überein.
\end{itemize}
\subsubsection*{Vielfachheiten}
Sei \f{\Phi\in} Hom\f{(V,V), \lambda\in} Spec\f{(\Phi)}. Dann sind Vielfachheiten wie folgt definiert:
\begin{itemize}
    \item \textbf{Geometrisch:} \f{\mu_g(\Phi,\lambda) :=} dim(Eig\f{(\Phi,\lambda)})
    \item \textbf{Algebraisch:} \f{\mu_a(\Phi,\lambda) :=} Häufigkeit der Nullstelle \f{\lambda} in \f{CP_\Phi(X)}
\end{itemize}
    \end{onehalfspace}

    

\end{document}